\documentclass[]{beamer}
%\usepackage[MeX]{polski}
%\usepackage[cp1250]{inputenc}
\usepackage{polski}
\usepackage[utf8]{inputenc}
\beamersetaveragebackground{blue!10}
\usetheme{Warsaw}
\usecolortheme[rgb={0.1,0.5,0.7}]{structure}
\usepackage{beamerthemesplit}
\usepackage{multirow}
\usepackage{multicol}
\usepackage{array}
\usepackage{graphicx}
\usepackage{enumerate}
\usepackage{amsmath} %pakiet matematyczny
\usepackage{amssymb} %pakiet dodatkowych symboli

\title{Przykladowy pusty Beamer}
\date{}

\begin{document}

\frame
{
	\maketitle
}
\frame
{
	\frametitle{Przykłady użycia bloków}
	\begin{block}
		{To jest zwykły blok}
		Tutaj umieścimy jego treść.
	\end{block}
	\begin{exampleblock}
		{To jest przykład}
		\texttt{for(int i=0;i<n;i++)
				cout<<"Hello world!";}
	\end{exampleblock}
	\begin{alertblock}
		{Uwaga!}
		Tu wpisujemy tekst, który wymaga podkreślenia.
	\end{alertblock}
}
\frame
{
\frametitle{Wypunktowania}
\begin{enumerate}
	\item<1-2> formatowanie tekstu i bibliografia
	\item<3-4> wzory matematyczne i pakiet algorithmic
	\item<2-3> tabele i obazki
	\item<3-4> pakiet beamer do tworzenia prezentacji
\end{enumerate}
}
\end{document}